\documentclass[arial,english,brazil,oneside]{estiloifes}

\usepackage[utf8]{inputenc}
\usepackage{lastpage}           % Usado pelo exemplo de ficha catalográfica
\usepackage[alf]{abntex2cite}
\usepackage{microtype}          % para melhorias de justificação
\usepackage{morefloats}         % permite mais floats
\usepackage{listings}
\usepackage{blindtext}          % para gerar texto aleatório (dummy text)
\usepackage{tikz}
\usetikzlibrary[topaths]
\usepackage{mathtools}
\usepackage{float}
\usepackage{setspace}
\usepackage{quoting}
\usepackage{setspace}% http://ctan.org/pkg/{setspace,lipsum}






%\setlist{parsep=\parskip,leftmargin=1.5cm}
\renewcommand{\baselinestretch}{1.5}
%%% Definição da linguagem padrão do documento (pacote babel) para
%%% definir que certas porções do texto (como o resumo, por exemplo)
%%% estão em uma língua estrangeira, usar as macros
%%% \foreignlanguage{languageB}{Text in another language}
%%% ou
%%% \begin{otherlanguage}{languageB}
%%% ...
%%% \end{otherlanguage}
\selectlanguage{brazil}


%%% Define que todos os códigos fontes construídos com o ambiente
%%% `lstlisting' terão uma borda simples.
\lstset{frame=single}


\newcommand{\ifestex}{\textsf{Ifes$7$}}

\titulo{Arquitetura publish/subscribe em dispositivos IoT}

\autor{Luan Grillo Silva}
\autorficha{Silva, Luan Grillo}

\orientador{PhD Rafael Silva Guimarães}


\instituicao{IFES - Instituto Federal do Espírito Santo}

\curso{Bacharelado em Sistemas de Informação}

\data{2022}

\local{Cachoeiro de Itapemirim - ES}

\preambulo{Trabalho de Conclusão de Curso apresentado à Coordenadoria
  do Curso de Sistemas de Informação do Instituto Federal do Espírito
  Santo, Campus Cachoeiro de Itapemirim, como requisito parcial para a obtenção do
  título de Bacharel em Sistemas de Informação.}

\tipotrabalho{Trabalho de Conclusão de Curso}


\begin{document}



\imprimircapa

\imprimirfolhaderosto*

%\begin{fichacatalografica}
  \vspace*{\fill}
  % Posição vertical
  \begin{center}
    % Minipage Centralizado
    \fbox{
      \begin{minipage}[t]{12.5cm}
        \vspace*{3mm}
        \begin{minipage}[t]{2cm}
          X000x
        \end{minipage}
        \begin{minipage}[t]{10cm}
          \imprimirautorficha.\vspace{2mm}

          \hspace{0.5cm}\imprimirtitulo\ / \imprimirautor. ---
          \imprimirlocal, \imprimirdata.\\ 

          \hspace{0.5cm}\pageref{LastPage} p.:\ il. (algumas color.);
          30 cm.\\ 

          \hspace{0.5cm}\imprimirorientadorRotulo~\imprimirorientador.\\

          \hspace{0.5cm}\imprimirtipotrabalho\ ---
          \imprimirinstituicao, \imprimirdata\\

          \hspace{0.5cm}        
          1. Palavra-chave1.
          2. Palavra-chave2.
          I. Orientador.
          II. Universidade xxx.
          III. Faculdade de xxx.
          IV. Título.
          
          \begin{flushright}
            CDU 00:000:000.0
          \end{flushright}
          \vspace*{1mm}
        \end{minipage}
      \end{minipage}
    }
  \end{center}
\end{fichacatalografica}

%
%%% ====================================================================
%%% Exemplo de construção da folha de aprovação. Depois da
%%% apresentação do trabalho, quando a folha de apresentação real
%%% tiver sido assinada pela banca, a folha real deve ser digitalizada
%%% e o ambiente abaixo deve ser substituído por:
%%% \includepdf{folhadeaprovacao_final.pdf}
\begin{folhadeaprovacao}
  \begin{center}
    {\ABNTEXchapterfont\MakeTextUppercase{\imprimirautor}}\\
    \begin{center}
      \ABNTEXchapterfont\MakeTextUppercase{\imprimirtitulo}\\
    \end{center}
    \hspace{.45\textwidth}
    \begin{minipage}{.5\textwidth}
      {\footnotesize{\imprimirpreambulo}}\\
    \end{minipage}%
  \end{center}
  \vspace{-1cm}%
  \begin{center}
    Aprovado em 20 de Novembro de 2017.\\[15mm]
    \textbf{COMISSÃO EXAMINADORA}\\[5mm]
    \assinatura{\imprimirorientador \\
      Instituto Federal do Espírito Santo - Cachoeiro de Itapemirim \\ Orientador}\vspace{-1cm}
    \assinatura{DSc. Convidado 1 \\
      Instituto Federal do Espírito Santo - Cachoeiro de Itapemirim}\vspace{-1cm}
    \assinatura{DSc. Convidado 2\\
      Instituto Federal do Espírito Santo - Cachoeiro de Itapemirim}\vspace{-1cm}
  \end{center}
\end{folhadeaprovacao}
%%%% --------------------------------------------------------------------
%%% Ambiente para escrita da declaração do autor
\begin{declaracaodoautor}

  \vspace*{1.5cm}

  Declaro, para fins de pesquisa acadêmica, didática e
  técnico-científica, que este Trabalho de Conclusão de Curso pode ser
  parcialmente utilizado, desde que se faça referência à fonte e ao
  autor.

  \vspace*{2.5cm}

  \centering

  \imprimirlocal, 20 de Novembro de 2017.

  \vspace*{2.5cm}

  \imprimirautor

  \vspace*{\fill}
  
\end{declaracaodoautor}

%%%% ====================================================================
%%% Ambiente para a escrita da dedicatória.
\begin{dedicatoria}
  \vspace*{\fill}
  \hspace{0.2\textwidth}
  \begin{minipage}{0.8\textwidth}
 Dedico esse trabalho primeiramente a Deus e a todos que de alguma forma contribuíram  para que o mesmo fosse realizado.
  \end{minipage}
\end{dedicatoria}
%%%% ====================================================================
%%% Agradecimentos
\begin{agradecimentos}
Primeiramente agradeço a Deus...

\end{agradecimentos}
%%%% ====================================================================
%%% Epígrafe
\begin{epigrafe}
  \vspace*{\fill}
  \begin{otherlanguage}{english}
    \begin{flushright}
      \begin{SingleSpace}
       "Boop, boop, beep, boop, beep".\\ 
        R2D2
      \end{SingleSpace}
    \end{flushright}
  \end{otherlanguage}
\end{epigrafe}
%%% Ambiente para resumo em português
\begin{resumo}
    Esse artigo aborda o desenvolvimento de arquiteturas distribuídas para uso em APIs de machine learning. Visando alta escalabilidade, baseia se principalmente na problemática de projetos que demandam alta quantidade de modelos computacionais de machine learning, buscando uma melhor distribuição compartilhada de todo o projeto. 

Palavras-chave:
Sistemas distribuídos, Machine Learning, API, Computação distribuída



\end{resumo}
%%% Ambiente para resumo em inglês.
\begin{resumo}[Abstract]
  \begin{otherlanguage}{english}
	  This article covers the development of distributed architectures for use in machine learning APIs. Aiming at high scalability, it is mainly based on the problem of projects that demand a high amount of computer models for machine learning, given a better shared distribution of the entire project.

Keywords:

Distributed Systems, Machine Learning, API, Distributed Computing
  \end{otherlanguage}
\end{resumo}

%\include{listaalgoritmos}
%%%% Lista de figuras
\renewcommand{\listfigurename}{Lista de figuras}
\pdfbookmark[0]{\listfigurename}{lof}
\listoffigures*
\cleardoublepage
%\include{listatabelas}
%%% Lista de quadros
\pdfbookmark[0]{\listadequadrosname}{loq}
\listadequadros*
\cleardoublepage
%%%% Lista de abreviaturas
\begin{abreviaturas}
\item Item 1
\item Item 2
\item Item 3
\item Item 4
\item Item 5

\end{abreviaturas}

%%%% Lista de símbolos
\begin{simbolos}
\simb{$\Gamma$} Letra grega Gama
\simb{$\Lambda$} Lambda
\simb{$\zeta$} Letra grega minúscula zeta
\simb{$\in$} Pertence
\simb{$\top$} Valor lógico máximo dentro de um reticulado regular
  booleano ou quasi-booleano.
\end{simbolos}
\cleardoublepage


%%% Sumário --- Table of Contents
\pdfbookmark[0]{\contentsname}{toc}
\tableofcontents*
\cleardoublepage

\mainmatter

% ----------------------------------------------------------------------

\chapter{\textbf{Introdução}} % Este comando é utilizado para criar capítulos

Com a exponencial de crescimento do poder de processamento dos computadores juntamente com a baixa do custo desses equipamentos forneceu um cenário ideal para a construção de redes extremamente complexas e descentralizadas. Segundo \cite{vanSteen1999}, o resultado da tecnologia atual é que agora se torna ainda mais possível uma arquitetura composta por inúmeros computadores em rede, de proporções inimagináveis.\par
Esse avanço possibilitou a conexão de todo o globo, encurtando distâncias, as redes de computadores distribuídas se tornaram um pilar essencial para a possibilidade de distribuição de uma complexa gama de serviços, hoje, fenômeno irreversível. Portanto, os sistemas distribuídos são conjunto de computadores independentes que se apresentam ao usuário como um sistema único e coerente \cite{vanSteen2016}. \par
A utilização de sistemas distribuídos nesta arquitetura será o ponto chave como objeto de estudo. Possibilitando a utilização de uma arquitetura distribuída, baseada em micro serviços, escalável, e que se comporte de forma suscetível a falhas.


\section{A problemática}
O intuito é fornecer ao final deste artigo um levantamento de referencial bibliográfico para construção de uma arquitetura distribuída de que contemple os seguintes requisitos, a escalabilidade, que define um projeto com capacidade de crescimento baseado na demanda e carga de trabalho, redundância, capacidade de ser flexível e tolerante a problemas, e uma boa relação custo-benefício no ambto de dispositivos de Internet das Coisas.\par 
A distribuição de recursos computacionais em aplicações que demandam baixos tempos de resposta é diretamente atrelado ao padrão, neste caso, o publish/subscribe utilizado e suas complexidades.\par


\iffalse
\section{Objetivo geral}


Texto de exemplo

\begin{itemize}
	\item Item 1
	
	\item Item 2
	
	\item Item 3
	
\end{itemize}

\begin{citacao}
	MAqui voce tem um exemplo de citação de mais de 3 linhas, o conteúdo do texto deve ser escrito em fonte menor, sem espaçamento e com recuo de 4cm em relacao ao texto padrão do trabalho.
\end{citacao}
\fi
% ----------------------------------------------------------------------

\chapter{\textbf{Sistemas distribuidos}} % Este comando é utilizado para criar capítulos

Com a exponencial de crescimento do poder de processamento dos computadores juntamente com a baixa do custo desses equipamentos forneceu um cenário ideal para a construção de redes extremamente complexas e descentralizadas. Segundo \cite{vanSteen1999}, o resultado da tecnologia atual é que agora se torna ainda mais possível uma arquitetura composta por inúmeros computadores em rede, de proporções inimagináveis.\par
Esse avanço possibilitou a conexão de todo o globo, encurtando distâncias, as redes de computadores distribuídas se tornaram um pilar essencial para a possibilidade de distribuição de uma complexa gama de serviços, hoje, fenômeno irreversível. \par
Portanto, os sistemas distribuídos são conjunto de computadores independentes que se apresentam ao usuário como um sistema único e coerente \cite{vanSteen2016}.\par
A utilização de sistemas distribuídos nesta arquitetura será o ponto chave como objeto de estudo. Possibilitando a utilização de clusters para processamento escalável da aplicação. Será analisado o impacto no uso de servidores cache para a distribuição dos modelos de machine learning de forma descentralizada aos clusters, o que possibilitará uma maior eficiência e custo-benefício final.\par



\chapter{\textbf{Padrão Publish/Subscribe}} % Este comando é utilizado para criar capítulos

A constante evolução da Internet das coisas (IoT) força, consequentemente, a evolução no contexto de redes e arquiteturas de computadores. A crescente de dispositivos IoT ligados à rede está aumentando exponencialmente ao decorrer dos anos, segundo \cite{7263372}, uma estimativa feita em 2015 para o número de dispositivos IoT implementados na internet é na ordem de 50 bilhões, em 2020, a estimativa é 1000 vezes em relação a 2015 de dispositivos IoT conectados. Isso só se torna possível com a implementação de novos padrões de arquiteturas de redes, que consequentemente, a possibilidade de implementar larga escala de dispositivos IoT, pois oferecem melhor custo benefício, eficiência energética se comparado a padrões antigos.\par

A utilização de abstrações mais naturais e simples é essencial para um contexto das redes dos dispositivos IoT. A ideia inicial de um padrão Publish/Subscribe é sugerida por \cite{Birman1987}, que visa uma implementação de redes em sistemas distribuídos de forma mais natural e que exista a possibilidade de alta escala.\par


Para um melhor entendimento do padrão publish/subscribe, primeiro precisamos definir o conceito largamente utilizado em soluções web, o Request-response. Segundo \cite{Luoto2018}, uma comunicação utilizando o padrão request-response, temos, em princípio dois atores, o cliente, qual irá requisitar os dados, e o servidor, que fará a parte de disponibilizar os recursos solicitados pelo cliente, como podemos observar na figura 1.\par



\begin{figure}[H]
    \centering
	\includegraphics[scale=0.95]{topics/reqres.png}
	\caption{Esquema de requisição-resposta de uma informação. Fonte: o autor.}
\end{figure}


Neste padrão, o servidor sempre aguarda as requisições que podem chegar dos clientes, entretanto, uma conexão é estabelecida temporariamente com o servidor, e após que as requisições forem respondidas, o pedido é dado concluído e não existe conexão posterior com servidor.\par
O padrão Observer também possui dois atores principais, os observadores (Observer) e sujeitos (Subject). Esses atores atuam diretamente na forma de exposição ao dado, qual, o observador, realiza uma requisição de inscrição em determinado sujeito, e dessa forma, ele passa a ser notificado quando existir alguma mudança de estado. O sujeito irá possuir uma lista com os observadores, possibilitando o envio das notificações somente quando houver alguma mudança no sujeito, observável na figura 2.\par

\begin{figure}[H]
    \centering
	\includegraphics[scale=0.98]{topics/observer.png}
	\caption{Publicação de notificação em esquema Observer. Fonte: o autor.}
	\label{fig:observer}
\end{figure}

Contudo, o padrão publish/subscribe, proposto por \cite{Birman1987}, possui semelhanças com o padrão de observador, neste caso,os atores semelhantes são o publicador (Publisher), responsável por publicar determinada informação em um tópico específico, o inscritos (Subscribe) que ingressa em determinado tópico que receberá notificações das publicações, esquematizado pela figura 3.\par

\begin{figure}[H]
    \centering
	\includegraphics[scale=0.90]{topics/pubsub.png}
	\caption{Esquema de publicação no padrão publish/subscribe. Fonte: o autor.}
	\label{fig:pubsub}
\end{figure}

Entretanto, o novo autor, o Broker, ficará responsável por manipular as notificações a serem enviadas do publicador ao inscritos. Dessa forma, os atores não precisam, necessariamente, terem um contato direto, ou ao menos se conhecer, pois tudo é intermediado pelo Broker, que fará a notificação de mudança dos estados e as informações a serem enviadas para os inscritos no tópico. O Publicador é responsável somente pelo envio das informação a ser distribuida, estabelecendo conexão única e direta ao Broker.


% ----------------------------------------------------------------------

\chapter{\textbf{Arquitetura baseada em Micro-serviços}} % Este comando é utilizado para criar capítulos

Com a crescente demanda de recursos computacionais distribuída, o enfrentamento com problemas de gestão e custo computacional  cresceram de forma acentuada, a forma de escalar os serviços de internet, em grande parte era feita de forma vertical (Scaling up), ou seja, aumentando a quantidade de recursos da máquina, como memória ou processadores para suprir determinada demanda. Pensando em uma forma mais eficiente de disponibilizar serviços na internet, surge o padrão de arquitetura baseada em micro-serviços, onde invés de existir um único e complexo serviço, as aplicações são desenvolvidas em partes, de maneira que cada serviço funcione de forma isolada, entretanto, para o usuário final, a percepção é que a aplicação é única. 

\begin{figure}[h!]
	\includegraphics[width=\linewidth]{topics/timeline.png}
	\caption{Linha do tempo das tecnológica do micro-serviços, por \cite{8354433}.}
	\label{fig:timelinemicroservices}
\end{figure}


O surgimento do termo micro-serviços é recente, segundo \cite{8354433}, o termo tem início em palestras sobre arquitetura de softwares, evoluindo até os dias atuais. A princípio, são inúmeras vantagens de utilização desse tipo de arquitetura, desde a melhor divisão de trabalho entre times, (cada equipe fica responsável por uma 'pequena' parte da aplicação), custo de infraestrutura reduzido (existe a possibilidade de escalar uma aplicação em demanda da utilização de forma mais coesa), eficiência energética e custo reduzido (poder computacional estará sendo escalado sob  demanda) e a possibilidade de escala horizontal (Scaling out, onde invés de incrementar recursos computacionais, é aumentado a quantidade de máquinas que realizam a tarefa).

%\include{topics/problematica}
%\include{topics/classificador}
% ----------------------------------------------------------------------

\chapter{\textbf{O protótipo}} % Este comando é utilizado para criar capítulos



\begin{figure}[h!]
	\includegraphics[width=\linewidth]{topics/sketch_bb.png}
	\caption{Diagrama eletronico do protótipo do dispositivo IoT. Fonte: autor}
	\label{fig:dataperformgraph}
\end{figure}

Os dispositivos de Internet das coisas devem se basear numa configuração mínima  ou superior a um microcontrolador ESP32, de baixo custo e proporciona o melhor entre custo benefício para protótipos em dispositivos de Internet das coisas, possui integrado funcionalidades de comunicação em rede Wireless, sendo primordial para os dispositivos a conexão de rede, o processador e memória são compatíveis para o projeto. Os principais pontos são seu espaço extremamente reduzido para a quantidade de funcionalidades presentes.\par
Neste projeto existem algumas formas de integração com o usuário, a primeira é o terminal RFID que proporciona a liberação da fechadura com um controle de acesso por usuário específico, ou seja, somente usuários autorizados previamente na plataforma conseguem destravar o mecanismo. A segunda iteração seria um botão interno para travamento ou liberação do mecanismo, com uma luz de indicação de status do mecanismo. A terceira será um dispositivo sonoro que em caso de problemas irá relatar o erro, como por exemplo problemas de energia ou de conexão.\par
A parte do mecanismo de travamento é composta por um motor que aciona um mecanismo linear de travamento, para determinar o estado do mecanismo, um sensor de efeito hall acusará se o mecanismo está travado ou destravado. Um outro sensor hall determinará o estado de abertura da porta.\par
Visando a redundância, o projeto contará com uma bateria para proporcionar em momentos de surto energético a possibilidade de utilização do mecanismo durante alguns dias. O desenvolvimento desse mecanismo também foi pensado em fornecer a possibilidade de utilização normal do equipamento para os usuários previamente listados na plataforma de controle do acesso, mesmo com surtos de conexão à internet.


\chapter{\textbf{Conclusão}} % Este comando é utilizado para criar capítulos
A arquitetura de uma aplicação de machine learning é composta por inúmeros desafios. Seu arranjo preserva os principais pilares dos sistemas distribuídos, como a disponibilidade, escalabilidade, confiabilidade e tolerância à falha, como descrito por \cite{vanSteen2016}. \par
Existirá grandes desafios durante a elaboração desta arquitetura, para isso, será fundamental o estudo e pesquisa de novas tecnologias, a comparação entre os principais tipos de soluções  nas quais irá auxiliar para a execução desta arquitetura.\par
Esta revisão foi fundamental para uma melhor estruturação e fundamentação  do projeto, e clareando os passos a serem seguidos para a conclusão da arquitetura distribuída seguindo os pilares de maior eficiência e benefícios possíveis.

%%% ====================================================================
%%% Início da parte pós-textual do documento.
\postextual


%%% Referências Bibliográfica

\bibliography{topics/bibliography}

%%%% Início dos apêndices ---------------------------------------------
\apendices


% --------------------------------------------------------------------
\chapter{Titulo}

\begin{enumerate}
    \item Item 1
    \item Item 2
\end{enumerate}

% --------------------------------------------------------------------



%%%% Início dos anexos ------------------------------------------------
\anexos

\partanexos*

\chapter{Exemplo de lombada}

\hspace*{0mm}\fbox{\includegraphics[scale=0.8]{pics/normas-ifes-apendice-d.png}}

\end{document}

