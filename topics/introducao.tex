% ----------------------------------------------------------------------

\chapter{\textbf{Introdução}} % Este comando é utilizado para criar capítulos

Com a exponencial de crescimento do poder de processamento dos computadores juntamente com a baixa do custo desses equipamentos forneceu um cenário ideal para a construção de redes extremamente complexas e descentralizadas. Segundo \cite{vanSteen1999}, o resultado da tecnologia atual é que agora se torna ainda mais possível uma arquitetura composta por inúmeros computadores em rede, de proporções inimagináveis.\par
Esse avanço possibilitou a conexão de todo o globo, encurtando distâncias, as redes de computadores distribuídas se tornaram um pilar essencial para a possibilidade de distribuição de uma complexa gama de serviços, hoje, fenômeno irreversível. Portanto, os sistemas distribuídos são conjunto de computadores independentes que se apresentam ao usuário como um sistema único e coerente \cite{vanSteen2016}. \par
A utilização de sistemas distribuídos nesta arquitetura será o ponto chave como objeto de estudo. Possibilitando a utilização de uma arquitetura distribuída, baseada em micro serviços, escalável, e que se comporte de forma suscetível a falhas.


\section{A problemática}
O intuito é fornecer ao final deste artigo um levantamento de referencial bibliográfico para construção de uma arquitetura distribuída de que contemple os seguintes requisitos, a escalabilidade, que define um projeto com capacidade de crescimento baseado na demanda e carga de trabalho, redundância, capacidade de ser flexível e tolerante a problemas, e uma boa relação custo-benefício no ambto de dispositivos de Internet das Coisas.\par 
A distribuição de recursos computacionais em aplicações que demandam baixos tempos de resposta é diretamente atrelado ao padrão, neste caso, o publish/subscribe utilizado e suas complexidades.\par


\iffalse
\section{Objetivo geral}


Texto de exemplo

\begin{itemize}
	\item Item 1
	
	\item Item 2
	
	\item Item 3
	
\end{itemize}

\begin{citacao}
	MAqui voce tem um exemplo de citação de mais de 3 linhas, o conteúdo do texto deve ser escrito em fonte menor, sem espaçamento e com recuo de 4cm em relacao ao texto padrão do trabalho.
\end{citacao}
\fi