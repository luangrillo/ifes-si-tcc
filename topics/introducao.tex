% ----------------------------------------------------------------------

\chapter{\textbf{Introdução}} % Este comando é utilizado para criar capítulos

A construção de um projeto de aprendizado de máquina envolve alta complexidade, cada projeto é único, incluindo seus requisitos. A busca por metodologias que melhorem as lacunas de uma arquitetura é essencial para grandes projetos, após a fase de prototipagem, a arquitetura geral do projeto deve ser pensada com cuidado, pois resultará diretamente no custo final de implantação. Para isso, uma pesquisa com uma forma distribuída e compartilhada de recursos para o projeto pode se tornar essencial para a execução de projetos.

\section{Motivação}

A motivação de pesquisa para esse tema surgiu durante a atuação em um projeto que utiliza um algoritmo de machine learning para fornecer uma  predição, aplicação disponibilizada a partir de uma API RESTful. O resultado da predição deve ser entregue com baixo tempo de resposta. A criação de uma arquitetura estável e escalável torna-se indispensável para o crescimento de grande parte dos projetos.\par
Na fase de expansão do projeto foi constatado um aumento do custo computacional em relação à sua expansão a novos clientes. O custo computacional tornou exponencial a quantidade de clientes atendidos, gerando grandes problemas, desde a dificuldade de escala do projeto, problemas de custo e baixa eficiência.\par
Durante a análise foi levantado que um dos principais problemas de escala desta aplicação é durante o armazenamento de estruturas de dados na memória. Os chamados modelos, são gerados a partir de uma etapa denominada treinamento, no qual uma grande quantidade de dados são processados por algoritmos de machine learning e, o retorno desse processamento é utilizado em para determinar a predição de certos tipos de informações.\par

\section{A problematica}
O intuito é fornecer ao final deste artigo um levantamento de referencial bibliográfico para construção de uma arquitetura distribuída de que contemple os seguintes requisitos, a escalabilidade, que define um projeto com capacidade de crescimento baseado na demanda e carga de trabalho, redundância, capacidade de ser flexível e tolerante a problemas, e uma boa relação custo-benefício.\par 
Os desafios de construção da arquitetura estão diretamente relacionados à proposta do projeto, em alguns casos, os recursos variam como processamento e memória, em alguns casos específicos, existe a necessidade de processadores gráficos para o processamento de alguns algoritmos de machine learning.\par
Em projetos que são compostos de inúmeros modelos a escalabilidade é afetada diretamente, pois exigem requerimentos mais altos para a execução do projeto. Visando a  solucionar esse problema, a criação de uma arquitetura que distribua esses modelos de forma descentralizada e compartilhada será a solução em estudo, o que poderá aumentar a eficiência desse ecossistema.\par
A distribuição de recursos computacionais em aplicações que demandam baixos tempos de resposta é diretamente atrelado ao algoritmo utilizado e suas complexidades. Suas excentricidades são desde o custo computacional, uso de memória primária para execução, uso da memória primária para armazenamento dos modelos  e tempo de execução.\par



\iffalse
\section{Objetivo geral}

Texto de exemplo

\begin{itemize}
	\item Item 1
	
	\item Item 2
	
	\item Item 3
	
\end{itemize}

\begin{citacao}
	MAqui voce tem um exemplo de citação de mais de 3 linhas, o conteúdo do texto deve ser escrito em fonte menor, sem espaçamento e com recuo de 4cm em relacao ao texto padrão do trabalho.
\end{citacao}
\fi