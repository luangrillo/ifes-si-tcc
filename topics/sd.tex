% ----------------------------------------------------------------------

\chapter{\textbf{Sistemas distribuidos}} % Este comando é utilizado para criar capítulos

Com a exponencial de crescimento do poder de processamento dos computadores juntamente com a baixa do custo desses equipamentos forneceu um cenário ideal para a construção de redes extremamente complexas e descentralizadas. Segundo \cite{vanSteen1999}, o resultado da tecnologia atual é que agora se torna ainda mais possível uma arquitetura composta por inúmeros computadores em rede, de proporções inimagináveis.\par
Esse avanço possibilitou a conexão de todo o globo, encurtando distâncias, as redes de computadores distribuídas se tornaram um pilar essencial para a possibilidade de distribuição de uma complexa gama de serviços, hoje, fenômeno irreversível. \par
Portanto, os sistemas distribuídos são conjunto de computadores independentes que se apresentam ao usuário como um sistema único e coerente \cite{vanSteen2016}.\par
A utilização de sistemas distribuídos nesta arquitetura será o ponto chave como objeto de estudo. Possibilitando a utilização de clusters para processamento escalável da aplicação. Será analisado o impacto no uso de servidores cache para a distribuição dos modelos de machine learning de forma descentralizada aos clusters, o que possibilitará uma maior eficiência e custo-benefício final.\par


